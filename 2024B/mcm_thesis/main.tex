%%%%%%%%%%%%%%%%%%%%%%%%%%%%%%%%%%%%%%%%%%%%%%%%%%%%%%%
%%%%%% 美国大学生数学建模竞赛(MCM/ICM)论文模板
%%%%%% 来源网站:www.latexstudio.net
%%%%%%%%%%%%%%%%%%%%%%%%%%%%%%%%%%%%%%%%%%%%%%%%%%%%%%%
%%%%%% 文件功能说明
%%% main.tex: 主LaTeX文件
%%% ref.bib: Bib 引用文献源文件
%%% mcmthesis.cls: LaTeX 格式文件
%%% code/: 附件代码文件夹
%%% figures/: 图片文件夹
%%% src/*.tex: 各个章节LaTeX 文档文件
%%%%%%%%%%%%%%%%%%%%%%%%%%%%%%%%%%%%%%%%%%%%%%%%%%%%%%%
%%%%%% 工具使用说明
%%% 公式图片转代码:https://snip.mathpix.com/home
%%% 流程图工具:
%%%%%%%%%%%%%%%%%%%%%%%%%%%%%%%%%%%%%%%%%%%%%%%%%%%%%%%



%%%%%%%%%%%%%%%%%%%%%%%%%%%%%%%%%%%%%%%%%%%%%%%%%%%%%%%
%%% 模板参数设置
%%%%%%%%%%%%%%%%%%%%%%%%%%%%%%%%%%%%%%%%%%%%%%%%%%%%%%%
\documentclass{mcmthesis}  % 文档类型
\mcmsetup{
        CTeX = false,   % 使用 CTeX 套装时,设置为 true
        tcn = 9555,   % 队伍控制号
        problem = B,  % 选题
        sheet = true,   % sheet页
        titleinsheet = true,   % sheet页显示标题
        keywordsinsheet = true,  % sheet页显示关键词
        titlepage = false,   % 标题页
        abstract = true  % 摘要
      }

%%%%%%%%%%%%%%%%%%%%%%%%%%%%%%%%%%%%%%%%%%%%%%%%%%%%%%%
%%% 导入宏包和引用文献源
%%%%%%%%%%%%%%%%%%%%%%%%%%%%%%%%%%%%%%%%%%%%%%%%%%%%%%%
\usepackage{lipsum}   % 仅用于生成示例文本

%%% Packages for Font
% \usepackage{palatino}  % 帕拉提诺体字体宏包
\usepackage{newtxtext} % Times New Roman
\usepackage{newtxmath} % 如果你需要Times New Roman风格的数学符号

%%% Packages for Referenece
\usepackage[hyperref=true,style=ieee]{biblatex}  % biblatex参考文献宏包
\addbibresource{ref.bib}  % 添加引用文献bib源

%%% Packages for Tables
\usepackage{booktabs} % Assumption 表格样式
\usepackage[table]{xcolor} % 加载 xcolor 包并允许表格颜色

%%% Packages for Graphs
\usepackage{graphicx} % 导入 graphicx 包以支持图片插入
\usepackage{subcaption} % 导入 subcaption 包以支持子图

%%% Packages for PseudoCode
\usepackage{amsmath} % 数学公式
\usepackage{algorithm} % 伪代码
\usepackage{algorithmic} % 伪代码

%%% Packages for Memo
\usepackage{multicol} % 导入 multicol 包以支持多列布局
\usepackage{lipsum}   % 仅用于生成示例文本
\usepackage{fancyhdr} % 导入 fancyhdr 包以自定义页眉页脚
\usepackage{geometry} % 导入 geometry 包以管理页面布局
\usepackage{eso-pic}  % 导入 eso-pic 包以添加背景图片
\usepackage{xcolor}   % 导入 xcolor 包以处理颜色
\usepackage{tikz}     % 导入 tikz 包以创建复杂背景


%%%%%%%%%%%%%%%%%%%%%%%%%%%%%%%%%%%%%%%%%%%%%%%%%%%%%%%%
%%%%%%%%%%%%%%%%%%%%%%%   首页   %%%%%%%%%%%%%%%%%%%%%%%
%%%%%%%%%%%%%%%%%%%%%%%%%%%%%%%%%%%%%%%%%%%%%%%%%%%%%%%%
\begin{document}  % 文档
\title{Submersible Rescue}  % 文章标题
\author{\small Team 9555}  % 作者,开启标题页才会显示
\date{\today}  % 日期,开启标题页才会显示



%%%%%%%%%%%%%%%%%%%%%%%%%%%%%%%%%%%%%%%%%%%%%%%%%%%%%%%%
%%%%%%%%%%%%%%%%%%%%%%%   摘要   %%%%%%%%%%%%%%%%%%%%%%%
%%%%%%%%%%%%%%%%%%%%%%%%%%%%%%%%%%%%%%%%%%%%%%%%%%%%%%%%
\begin{abstract}  % 摘要

	The 2023 Wimbledon \cite{vaswani2017attention} Gentlemen’s final witnessed a wonderful batter, a Spanish player ended great tennis player Novak Djokovic's Grand Slam. Not only did the match see a rising star apparence, but also revealed the crucial role of ``momentum" in tennis. In this regard, based on relevant data, we conduct in-deep studies of ``momentum" in tennis.

	Firstly, we did the data cleaning. We determined the cumulative score ratio of a particular game as the label vector, then fused the physical time-series feature and the temporal features of the match phase as the input vector. After processing the data, we chose the LSTM model with the RNN framework as the basic framework of the model. Considering its objective position in the overall game, we used its variant BiLSTM. At the same time, we introduced residual structure, L2 regularization, and the dropout mechanism to avoid the overfitting problem and stacked a 3-layer network to increase the expressive power. We used a deep learning approach to train the model and visualized the results. By comparing the real data and the predicted data, we found our model can well capture the rapidly changing situation in the field. We calculated the probability of "winning under one's serve" as an indicator of a player's performance and compared it with a player’s probability. This method can quantify player performance and remove the effect of factors like servers.

	For correlation between ``momentum" and ``swings" analysis, we quantify ``momentum". To construct a quantitative model, we use the physical definition and formula for ``momentum''$P=mv$ for reference and construct the initial model. Next, with the purpose of  making our model in line with the actual situation, we constantly try introducing parameters to optimize the initial model for the most suitable model.

	Based on the final model, we denote ``swings" and make it harmonize with `` momentum".  Owing to the ternary discrete nature of ``swings"  data, we adopt  image analysis to explore the correlation between ``momentum" and ``swings".  After analyzing corresponding images and comparing with actual match scores, we conclude that there is indeed a correlation between ``momentum" and ``swings" and it may even determine the direction of the game to a certain extent.

	We determined which places are the nodes where the momentum shifts. Since we took an approach similar to smoothing relative momentum, we only need to look at the places where the momentum crosses the zero point or drops off sharply. We looked at all the labels in the dataset, coded the features, such as one-hot coding, and numericalized some of the symbols to get some computable vectors of information about the features and used them as inputs to fit a simple multiple regression model to the "momentum". After the training, we sorted the features, and then analyzed them one by one to give suggestions to the athletes using these features.


	\begin{keywords}  % 关键词
		AHP; SAR(submersible search and rescue); Monte Carlo sampling; Runge-Kuta; Sensitivity Analysis.
	\end{keywords}  % 结束关键词
\end{abstract}  % 结束摘要

\maketitle  % 生成sheet页


%%%%%%%%%%%%%%%%%%%%%%%%%%%%%%%%%%%%%%%%%%%%%%%%%%%%%%%%
%%%%%%%%%%%%%%%%%%%%%%%   目录   %%%%%%%%%%%%%%%%%%%%%%%
%%%%%%%%%%%%%%%%%%%%%%%%%%%%%%%%%%%%%%%%%%%%%%%%%%%%%%%%
\begingroup
\setlength{\parskip}{0em}
\tableofcontents  % 生成目录表
\endgroup

%%%%%%%%%%%%%%%%%%%%%%%%%%%%%%%%%%%%%%%%%%%%%%%%%%%%%%%%
%%%%%%%%%%%%%%%%%%%%%%%   正文   %%%%%%%%%%%%%%%%%%%%%%%
%%%%%%%%%%%%%%%%%%%%%%%%%%%%%%%%%%%%%%%%%%%%%%%%%%%%%%%%
\newpage  % 开始新的一页

\section{Introduction}  % 一级标题
\
\indent According to historical research, the reservoir was first built around 600 BC. With the devel-opment of science and technology, the function of reservoirs has gradually expanded to irrigating farmland, supplying domestic water, preventing floods and developing hydropower to help maintain people’s normal life.

According to historical research, the reservoir was first built around 600 BC. With the devel-opment of science and technology, the function of reservoirs has gradually expanded to irrigating farmland, supplying domestic water, preventing floods and developing hydropower to help maintain people’s normal life.

\subsection{Background}
\subsection{Problem Restatement and Analysis}
\subsection{Overview of our work}

\section{Assumptions and Notaions}


\subsection{List of Notaion} % Notaion Table


\begin{table}[ht]
	\centering
	\caption{Symbols and explanations}
	\begin{tabular*}{\textwidth}{@{} c @{\extracolsep{\fill}} c c @{}}
		\toprule
		\rowcolor{white}
		\textbf{Symbols} & \textbf{Description} &  \textbf{Unit}   \\
		\midrule
		$S_{A/B}(i)$        & The total score won by player $A/B$ at time step $i$      & $m/s$ \\
		$\theta_{A/B}$      & The ability of player $A/B$ to maintain ``momentum"       & $m/s$  \\
		$M_{A}(i)$          & The relative ``momentum" of $A$ at the time step $i$      & $s$    \\
		$J$                 & Loss function for the multiple regression model           & $m/s$  \\
		$\alpha$            & Multiple regression weights for each feature              & $m/s$  \\
		\bottomrule
	\end{tabular*}
\end{table}


\subsection{Assumptions} % Assumptions

To simplify the problem and make it convenient for us to simulate real-life conditions, we make the following basic assumptions, each of which is properly justified.

\begin{itemize}
	\item \textbf{Assumption 1:} The sea surface of the Ionian Sea remained stable during our study period. \newline
	      \textbf{Justification:} The Ionian Sea is located in the central Mediterranean Sea, bordered by the Adriatic Sea to the north, Calabria and Sicily in Italy to the west, and Albania as well as many Greek islands to the east, making it a vast area. Therefore we assume that the sea area will not change significantly and that there is no risk of the submersible and the host ship running aground.


	\item \textbf{Assumption 2:} Climate types in the Ionian Sea remained constant over the study period.  \newline
	      \textbf{Justification:} The climate type of the Ionian Sea is Mediterranean, influenced by the subtropical high and the westerly wind belt, the salinity of the seawater, the air pressure, and the seawater temperature are roughly stable, so that the size of the wind, the strength of the ocean currents and the density of the seawater are approximately the same.


	\item \textbf{Assumption 3:} The Ionian Sea will not experience large natural disasters during the study period. \newline
	      \textbf{Justification:} Large natural disasters such as earthquakes, tsunamis, and typhoons can
\end{itemize}




\section{Model I: Randomized roaming models and equipment selection}

\section{Model II: AHP-based Rescue Equipment Selection Model}

\[
	\begin{aligned}
		 & P(0,0,0)=1, i=0, j=0                                                                                \\
		 & P(i, 0, k)=P(i-1,0, k-1) \times \frac{1}{3}, j=0, i \geq 1                                          \\
		 & P(0, j, k)=P(i, j-1, k-1) \times \frac{1}{3}, i=0, j \geq 1                                         \\
		 & P(i, j, k)=[P(i-1, j, k-1)+P(i, j-1, k-1)+P(i-1, j-1, k-1)] \times \frac{1}{3} ; i \geq 1, j \geq 1
	\end{aligned}
\]

\section{Model III: Rescue strategy and survival probability model}

\subsection{Calculation of Ocean Current Velocity at Corresponding Depth}
\
\indent In force equilibrium:

\begin{equation}
	M_{\text{sub}} g = \rho(z_{\text{eq}}) g V_{\text{sub}}
\end{equation}

Solving for \( z_{\text{eq}} \):

\begin{equation}
	z_{\text{eq}} = \rho^{-1}\left(\frac{M_{\text{sub}}}{V_{\text{sub}}}\right)
\end{equation}

To solve for \( z_{\text{eq}} \), methods such as graphical methods, bisection method, or Newton’s method can be used, as the function is monotonic in the range \( 30m \sim 200m \). The following graph illustrates the density as a function of depth.

%% The Newton's Method to solve z by given M & V
\begin{algorithm}[H]
	\caption{Fixed Point Iteration to Solve \( \rho(z) - \frac{M}{V} = 0 \)}
	\begin{algorithmic}[1]
		\STATE \textbf{Input:} Tolerance \( \epsilon \), Maximum iterations \( n_{\text{iter}} \), known constants \( M, V \)
		\STATE \textbf{Output:} Solution \( z_{\text{eq}} \)

		\STATE \( n \leftarrow 0 \)
		\STATE \( z \leftarrow z_0 \sim \mathcal{U}(30, 200) \)

		\WHILE{\( \left| z_{\text{new}} - z \right| > \epsilon \) \AND \( n < n_{\text{iter}} \)}
		\STATE \( z_{\text{new}} \leftarrow \rho(z) + z - \frac{M}{V} \)
		\STATE \( z \leftarrow z_{\text{new}} \)
		\STATE \( n \leftarrow n + 1 \)
		\ENDWHILE

		\STATE \textbf{Return} \( z \)
	\end{algorithmic}
\end{algorithm}


Then, the ocean current velocity can be calculated using the formula:

\begin{equation}
	v(z_{\text{eq}}) = v(z_0) \cdot e^{-\frac{z_{\text{eq}} - z_0}{\delta}}
\end{equation}

Where \( \delta = \sqrt{\frac{v}{\Omega \sin(\phi)}} \).

In this case, we use \( \delta = 112m \) (assumed without questioning the reasoning behind it).


\subsection{Calculating the Drift Distance of the Submersible During One Search Round}
\
\indent Assuming the ship quickly reaches a steady drift state, the speed of the submersible’s drift is the ocean current velocity. The distance the submersible will drift during one search period is:

\[
	s = t_{\text{search}} \cdot v(z_{\text{eq}})
\]

Typically, this distance \( s \) is several times the length \( d \), and the number of steps \( \Delta n \) is:

\[
	\Delta n = \lceil \frac{s}{d} \rceil
\]

This means: within one search cycle, the object might move several grid steps.


\subsection{Search process}
\
\indent since we assume the time it takes for the rescue equipment to return to the starting search location is much smaller than the search time, we establish a grid starting from the initial loss of contact location at \( (0, 0, 0) \), where the last 0 represents the step.

ignoring seasonal variations, we start with the assumption that the drifting probability for a grid point is:

\[
	\begin{bmatrix}
		0 & \frac{1}{3} & \frac{1}{3} \\
		0 & 0           & \frac{1}{3} \\
		0 & 0           & 0           \\
	\end{bmatrix}
\]

\noindent at step \( k \), the probability of finding the lost submersible at a given location is represented as:

\[
	p(i,j,k)
\]

\noindent the state transition equations are:

\[
	p(i,j,k) =
	\begin{cases}
		1,                                                                              & \text{if } i=0, j=0, k=0                \\
		p(i-1,0,k-1) \times \frac{1}{3},                                                & \text{if } j=0, i \geq 1, k \geq 1      \\
		p(0,j-1,k-1) \times \frac{1}{3},                                                & \text{if } i=0, j \geq 1, k \geq 1      \\
		\left[ p(i-1,j,k-1) + p(i,j-1,k-1) + p(i-1,j-1,k-1) \right] \times \frac{1}{3}, & \text{if } i \geq 1, j \geq 1, k \geq 1
	\end{cases}
\]

\noindent we calculate the probability matrix \( p[:,:,T_0] \), which represents the probability of finding the submersible at different locations at time \( t \).

next, we find the coordinates corresponding to the maximum probability:

\[
	i^*, j^* \leftarrow \mathop{\arg\max}\limits_{i,j} p(i,j,T_0)
\]

\noindent then, we set:

\[
	p(i^*, j^*, T_0) \leftarrow 0
\]

this indicates that the submersible is not found at that location. we update the matrix \( p[:,:,T_0+\Delta n] \) and repeat the process.

thus, by the \( k \)-th search round, the probability of finding the submersible is:

\[
	p(k) = \mathop{\max}\limits_{i,j} p(i,j,T_0+k\Delta n)
\]

we can compute this for multiple values of \( k = 1, 2, \dots, 20 \) and store the results in an array.

Additionally, we can compare this strategy with a random search approach. in the random search, the probability of finding the submersible at time \( k \) is:

\[
	p(k) = \frac{1}{(T_0+k\Delta n)^2}
\]

\noindent finally, we can generate a comparison plot between the two strategies.



\section{Sensitivity Analysis}

\section{Strengths and Weaknesses}

\subsection{Strengths}
\subsection{Weaknesses}



%%% Referenece
\begingroup
\setlength{\parskip}{0em}
\newpage
\printbibliography[heading=bibintoc]  % 打印引用文献列表
\endgroup


%%%%%%%%%%%%%%%%%%%%%%%%%%%%%%%%%%%%%%%%%%%%%%%%%%%%%%%%
%%%%%%%%%%%%%%%%%%%%%%%   Memo   %%%%%%%%%%%%%%%%%%%%%%%
%%%%%%%%%%%%%%%%%%%%%%%%%%%%%%%%%%%%%%%%%%%%%%%%%%%%%%%%
\clearpage
\phantomsection % 为 hyperref 准备锚点
\addcontentsline{toc}{section}{Memorandum} % 添加到目录
\begin{multicols}{2}
	\section*{Memo for Greek Government}
	\lipsum[1-4] % 示例文本
	\lipsum[5-8]
\end{multicols}


% \begin{appendices}  % 附录
% \section{First appendix}  % 一级标题
% Here are simulation programmes we used in our model as follow.\\
% \textbf{C++ source code:}
% \lstinputlisting[language=C++]{./code/main.cpp}
% \section{Second appendix}  % 一级标题
% \textbf{Python source code:}
% \lstinputlisting[language=python]{./code/python.py}
% \end{appendices}  % 附录结束


\end{document}  % 文档结束
